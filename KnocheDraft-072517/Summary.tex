
\section{Conclusions}

LUX's WS2014-16 data is complicated by a nonuniform electric field in the detector.  The variation of the electric field in space and time produces variation in the recombination of S1 and S2 events as a function of energy, time, space, and recoil type.  If this recombination variation is not properly separated from detector inefficiency effects in $^{83m}$Kr data, the KrypCal corrections produce data which has poor energy reconstruction with unreasonably high extraction efficiency estimates, worsened energy resolution, and widened ER bands.  We have developed multiple methods to relate the strength of the field effect in S1 and S2 data to the $^{83m}$Kr S1a/S1b ratio.  These two relationships can be used to separate the field effects from detector inefficiency effects prior to producing KrypCal corrections from $^{83m}$Kr calibrations taken at any point in time.  This process results in better energy reconstruction with g1 and extraction efficiency values close to our expectations,  improved energy resolution, and improved ER band width.  However, since the field effects remain in the corrected data, a spatial and time dependence remain in the corrected S1 and S2 signal, leading to complications in calibrating the corrected ER band over time.

